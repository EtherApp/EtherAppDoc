\section{Systembeschreibung} % (fold)
\label{sec:system}

\subsection{ShowPad}
\label{sub:system:ShowPad}

Das ShowPad\footnote{\url{http://pad.shownot.es}} ist ein Webservice und Teil des Shownotes-Projektes\footnote{\url{http://shownot.es}}.
Es dient dem gemeinsamen und gleichzeitigen Verfassen von Sendungsnotizen, auch Shownotes genannt, zu Podcasts. Dies Funktion wird durch ein, in den Webservice integriertes, EtherPad\footnote{\url{http://etherpad.org/}} bereit gestellt. 
Hierbei handelt es sich im Wesentlichen um einen kollaborativen Online Editor, welcher für das Schreiben von Sendungsnotizen mittles Erweiterungen angepasst wurde.
Einem registrierten Benutzer bietet das ShowPad die Möglichkeit neue Dokumente (Pads) für Sendungsnotizen, über die Hauptseite, anzulegen oder auf ein bereits existierendes Dokument zuzugreifen.  

\subsection{Shownotes Message Service}
\label{sub:system-sms}

Der Shownotes-Message-Service (SMS) \footnote{\url{https://github.com/shownotes/shownotes-message-service}} ist ein Notification Service für Jabber/XMPP-Clients.
Er informiert angemeldete Benutzer, auf deren Jabberaccounts, über anstehende Live-Podcasts, sofern ein Benutzer den entsprechenden Podcast vorher abonniert hat.
Des Weiteren kann ein BeNntzer über die Erstellung eines Pads im ShowPad informiert werden.
Die Registrierung und die anschließende Aktivierung von Notifications erfolgt direkt durch den Benutzer via Jabber/XMPP-Client.
Eine weitere Möglichkeit bietet eine REST-API, welche die gleichen Einstellungen über einen externen Service zulässt.

\subsection{Hörsuppe.de}
\label{sub:system:Hoersuppe}

Die \glqq Hörsuppe\grqq{ }\footnote{\url{http://hoersuppe.de}} ist ein Onlinekalender für Podcasts.
Nutzer können hier aktuelle oder zukünftige Sendetermine von Podcasts abrufen.
Webservices bietet die Seite eine REST-API, welche sowohl dem ShowPad als auch dem SMS als Datenbasis dient\citeque{IEEE802-11}.

% section system (end)



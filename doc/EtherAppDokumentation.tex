%%This is a very basic article template.
%%There is just one section and two subsections.
\documentclass[
    numbers=noenddot, %Keine Punkte am Ende des TOC
    toc=flat, %Flache TOC
    12pt, % Schriftgröße 
    titlepage, % es wird eine Titelseite verwendet 
    %parskip=half, % Abstand zwischen Absätzen (halbe Zeile) 
    listof=totoc, % Verzeichnisse im Inhaltsverzeichnis aufführen 
    %^enlargefirstpage,
    bibliography=totoc, % Literaturverzeichnis im Inhaltsverzeichnis aufführen 
    %index=totoc, % Index im Inhaltsverzeichnis aufführen 
    %captions=tableheading, % Beschriftung von Tabellen unterhalb ausgeben 
    %draft % Status des Dokuments (final/draft) draft hinzufügen zum anziegen der zeilen ende
]{scrartcl} 

\usepackage[utf8]{inputenc} %UTF8 nutzen
\usepackage[ngerman]{babel} %deutsch umlaute

\usepackage[a4paper,left=2.5cm,right=2.5cm, top=2cm, bottom=2cm]{geometry}
%Zeilenabstand anpassen
\usepackage{setspace}
%\onehalfspacing
%Neues Seiten format, bitte
%\renewcommand{\geometry}{left=3cm,right=2.5cm, top=2.5cm, bottom=2.5cm}
\setlength{\footskip}{0.7cm}

\usepackage{fancyhdr} %Huebsche footer und header
\usepackage[bookmarks=true]{hyperref} %fuer klickbare links und url formatierungen
\hypersetup{pdfborder={0 0 0}} %--rahmen um links in pdf ausschalten

\usepackage{bibgerm}
%\usepackage[superscript]{cite} %Hochgestellte Zahlen für quellen

%Verzeichnisse
\usepackage[resetlabels]{multibib}
\newcites{que}{Quellenverzeichnis}

\usepackage{listings} %für quellcode

\usepackage{graphicx} %grafiken einbinden
%\usepackage{warpfig} % Text um Bild

\usepackage{multirow}
\usepackage{amssymb}
\usepackage{MnSymbol}
%Beispiele fuer manuelle Silbentrennung
% Silbentrennung direkt mit /- oder /"
%\hyphenation{ge-wuensch-ten} 

% include pdfs directly into tex
\usepackage{pdfpages}

% LISTINGS
\usepackage{color} % Wird benoetigt fuer listing

\definecolor{dkgreen}{rgb}{0,0.6,0}
\definecolor{gray}{rgb}{0.5,0.5,0.5}
\definecolor{mauve}{rgb}{0.58,0,0.82}

\usepackage{listings} %fuer listings von quellcode
\lstset{
	numbers=left,
	numberfirstline=false,
  xleftmargin=1.5cm,
  xrightmargin=0.5cm, 
	numberstyle=\tiny,
	numbersep=5pt,
	frame=single,
	basicstyle=\footnotesize\ttfamily,
	breaklines=true,
	showstringspaces=false,
	commentstyle=\color{gray},
	% keywordstyle=\color{black},
	postbreak=\raisebox{0ex}[0ex][0ex]{\ensuremath{\rcurvearrowse\space}}
}

\lstdefinestyle{customhtml}{
  numbers=left,
  numberfirstline=false,
  belowcaptionskip=1\baselineskip,
  breaklines=true,
  frame=single,
  xleftmargin=1.5cm,
  xrightmargin=0.5cm,
  language=PHP,
  showstringspaces=false,
  basicstyle=\scriptsize\ttfamily,
  keywordstyle=\bfseries\color{mauve},
  commentstyle=\itshape\color{blue},
  identifierstyle=\color{mauve},
  stringstyle=\color{dkgreen},
  numberstyle=\ttfamily
}

%Tiefe der Kapitel und Unterkapitel
%-1	keine Überschrift
%0	Kapitelüberschriften
%1	Kapitel- und Abschnittsüberschriften
%2	Kapitel- bis Unterabschnittsüberschriften
%3	Kapitel- bis Unterunterabschnittsüberschriften
%4	Kapitel- bis Paragraphsüberschriften
%5	alle Überschriften
%\setcounter{secnumdepth}{2}
%\setcounter{tocdepth}{3}


\begin{document}

    %\section{Deckblatt}
\thispagestyle{empty}
\begin{center}
\Large{Hochschule für Telekommunikation Leipzig (FH)}\\

\Large{Profilierung Mobile Applikationen}\\
\end{center}
\begin{verbatim}


\end{verbatim}
\begin{center}
\textbf{\Huge{Dokumentation EtherApp}}
\end{center}
\begin{verbatim}








\end{verbatim}
\begin{flushleft}
\begin{tabular}{llp{12cm}}
\textbf{Thema:} & & \LARGE{Entwicklung einer Serviceapp für ein Etherpad Light}\\
\end{tabular}
\end{flushleft}
\begin{verbatim}
















\end{verbatim}
\begin{flushleft}
\begin{tabular}{lll}
& & \\
& & \\
\textbf{Vorgelegt von:} & & Ferdinand Malcher \\
                        & & Martin Stoffers \\
& & \\
& & \\
& & \\
\textbf{Betreuer:} & & Herr Prof. Dr. Ulf Schemmert
\end{tabular}
\end{flushleft}
%\includegraphics[scale=0.25,clip=false]{Logo/logo.pdf}

    \newpage

    \pagestyle{fancy} %eigener Seitenstil an hier

    %Weitere Möglichkeiten    
    %\fancyhead[L]{Titel} %Kopfzeile links
    %\fancyhead[C]{} %zentrierte Kopfzeile
    %\fancyhead[R]{Name} %Kopfzeile rechts
    %\fancyfoot[C]{\thepage} %Seitennummer
    %\renewcommand{\headheight}{20.49998pt} 
    \fancyhf{} %alle Kopf- und Fußzeilenfelder bereinigen
    \renewcommand{\headrulewidth}{0.0pt} %obere Trennlinie unsichtbar
    \renewcommand{\footrulewidth}{0.0pt} %untere Trennlinie unsichtbar



    \clearpage % Seitenende erzwingen und formatierungen auf standart

	\pagestyle{fancy} 
    \rhead{\thepage} \chead{} \lhead{Dokumentation EtherApp} 
    \lfoot{Ferdinand Malcher, Martin Stoffers} \cfoot{} \rfoot{\today}
    \renewcommand{\headrulewidth}{0.4pt} %obere Trennlinie sichtbar setzen
    \renewcommand{\footrulewidth}{0.4pt} %untere Trennlinie sichtbar setzen

    %\setcounter{page}{1}	

    \tableofcontents
    \newpage % Neue Seite

    \section*{Abkürzungsverzeichnis}
\label{abk:Abkuerzungsverzeichnis}
\begin{tabbing}
\hspace*{3,5cm}\=\kill

API \> Application Programming Interface\\ \\

EPL \> EtherPad Lite\\ \\

HTTP \> HyperText Transfer Protocol\\ \\

JSON \> JavaScript Object Notation\\ \\

SDK \> Software Development Kit\\ \\

SQL \> Structured Query Language\\ \\

UI \> User Interface, Benutzeroberfläche\\ \\

URL \> Uniform Resource Locator\\ \\

UUID \> Universally Unique Identifier\\ \\
\end{tabbing}

    \newpage

    \parindent0pt % Keine Absatzeinrückungen
    \parskip2ex %Absatzhöhe
    \setlength{\abovedisplayskip}{10pt}

    \section{Systembeschreibung} % (fold)
\label{sec:system}

\subsection{Etherpad Lite}
\label{sub:system:etherpad}

\textit{Etherpad Lite}\citeque{etherpad} ist ein kollaborativer Online-Editor.
Anwender können Textdokumente, sogenannte Pads, gemeinschaftlich und in Echtzeit online bearbeiten.
Die Oberfläche ist browserbasiert.
Innerhalb der Pads ist durch farbliche Markierung gekennzeichnet, welche Bearbeitungen von welchem Nutzer vorgenommen wurden.
Eine Instanz des Etherpad Lite ist standardmäßig offen.
Damit kann jeder Anwender Pads anlegen und bearbeiten.
Das System speichert Revisionen des Pad-Inhalts, die später abgerufen werden können.
Die Oberfläche bietet außerdem die Möglichkeit zum Export des Inhalts in verschiedene Dokumentformate sowie einen Onlinechat zur Verständigung unter den Autoren.\\
Das Projekt ist Open-Source und damit quelloffen und frei für jeden.

\subsection{Abgrenzung und Zielstellung}
Die Oberfläche des Etherpad Lite ist schlicht und übersichtlich gehalten.
Sie bietet benutzerrelevante Funktionen, die sich auf ein einzelnes Pad beziehen.
Administrative Funktionen, die bestimmten Benutzerkreisen vorbehalten bleiben müssen, sind auf der Oberfläche nicht implementiert.
Es gibt somit keine Möglichkeit, z.B. alle verfügbaren Pads anzuzeigen oder einzelne Pads zu löschen.\\
Es existiert eine umfangreiche HTTP-API, über die diverse Funktionen verfügbar gemacht werden.
Für die praktische Verwendung dieser API ist ein Frontend sinnvoll.\\
Das Projekt EtherApp hat zum Ziel, einen Teil der API-Funktionen abzubilden und eine App zur Administration verschiedener Etherpad-Lite-Instanzen zu entwickeln.
Das beinhaltet insbesondere folgende Funktionen:

\begin{itemize}
	\item Anzeige aller Pads
	\item Anzeige von Meta-Informationen für einzelne Pads (Benutzer online, Revisionen, Datum, …)
	\item Anlegen neuer Pads
	\item Löschen von Pads
	\item Anzeige des Pad-Inhalts
	\item Teilen der Pad-URL über soziale Dienste und E-Mail
	\item Rücksetzen des Inhalts auf ältere Revision
	\item Anzeige und Verwaltung von Gruppen
	\item Verwaltung mehrerer EPL-Instanzen und Profilverwaltung
\end{itemize}

% section system (end)

\section{Vorgehen} % (fold)

\subsection{Zugriff auf die HTTP-API}
Das EPL bietet von Haus aus eine umfangreiche HTTP-API\citeque{epl-httpapi}.
Sie stellt eine Reihe von Funktionen für administrative Aufgaben bereit, die über die offene Weboberfläche nicht zur Verfügung stehen.
Der Zugriff auf die API ist nur mit einem Shared Secret (API Key) möglich, der vom Administrator festgelegt werden muss.\\
Für den Zugriff auf die API existieren bereits Java-Bibliotheken, die alle API-Funktionen auf Methoden einer Java-Klasse abbilden.
Für das Projekt EtherApp kam dabei die Implementation\citeque{hollinger} des US-amerikanischen Entwicklers Jordan Hollinger zum Einsatz.\\
Die Bibliothek beinhaltet eine Klasse \texttt{EPLiteClient}, die mit den Zugangsdaten für die EPL-API initialisiert wird.
Mit einem Objekt dieser Client-Klasse können Anfragen auf die jeweilige API durchgeführt werden, die als Java-Datenstruktur (\texttt{String} oder \texttt{HashMap}) zurückgegeben werden.

\subsection{Selbst definierte Listen}
Das Android-SDK stellt mit dem \texttt{ListView} ein View-Element zur Verfügung, mit dem Elemente in einer scrollbaren Liste angezeigt werden können.\\
Zur Anpassung von Daten und das tatsächliche Darstellen in der Liste wird ein Adapter benötigt.
Android stellt bereits eine Reihe von Adaptern zur Verfügung (\texttt{BaseAdapter}, \texttt{ArrayAdapter}, ...).
Deren Nachteil ist jedoch eine festgelegte Darstellungsweise; es ist auf einem Listenelement lediglich ein \texttt{TextView} dargestellt.\\
In unserem Anwendungsfall sollten in den Listenelementen der Padliste neben dem eigentlichen Pad-Namen zusätzliche Statusinformationen zum Pad sowie ein Löschen-Button angezeigt werden.
Dieses Ziel konnte nur mit einem eigenen Adapter\footnote{\texttt{de.etherapp.adapters.PadlistBaseAdapter.java}} und einem selbst gestalteten Listenelement\footnote{\texttt{de.etherapp.beans.PadlistItem.java}} erreicht werden.\\


\subsection{Asynchrones Laden der Padliste}
Um eine komplette Padliste inklusive der Metadaten zu jedem Pad abzurufen, sind folgende Anfragen an die HTTP-API nötig:
\begin{itemize}
	\item Abrufen der Padliste (nur IDs)
	\item Abrufen der Metadaten zu jedem einzelnen Pad
		\begin{itemize}
			\item Anzahl der User online
			\item Anzahl der Revisionen
			\item Datum der letzten Bearbeitung
		\end{itemize}
\end{itemize}

Für jedes Pad sind also drei HTTP-Anfragen abzusetzen.
Bei einem Volumen von 300 Pads sind damit 901 HTTP-Anfragen nötig, um alle Daten der Padliste abzurufen, auch wenn der Benutzer diese zunächst gar nicht benötigt!
Dieser Umstand erfordert eine Lösung, mit der die Inhalte eines Listenelements erst dann geladen werden, wenn das Element auch angezeigt wird.
Das Nachladen sollte ansynchron erfolgen.\\
 % Inhalt einfuegen!!!!!!!11ELF

    \begin{flushleft}% Flattersatz
      \begin{appendix}

        \fancyhf{} %alle Kopf- und Fußzeilenfelder bereinigen
        \renewcommand{\headrulewidth}{0.0pt} %obere Trennlinie unsichtbar
        \renewcommand{\footrulewidth}{0.0pt} %untere Trennlinie unsichtbar

        \newpage
        \label{que:Quellenverzeichnis}
        \bibliographystyleque{gerunsrt}
        \bibliographyque{quellen}

      \end{appendix}

    \end{flushleft}

\end{document}

